
\documentclass[english]{beamer}
\usepackage{lmodern}
\renewcommand{\sfdefault}{lmss}
\renewcommand{\ttdefault}{lmtt}
\usepackage[T1]{fontenc}
\usepackage{amsmath}
\usepackage{amssymb}

\makeatletter
%%%%%%%%%%%%%%%%%%%%%%%%%%%%%% Textclass specific LaTeX commands.
% this default might be overridden by plain title style
\newcommand\makebeamertitle{\frame{\maketitle}}%
% (ERT) argument for the TOC
\AtBeginDocument{%
  \let\origtableofcontents=\tableofcontents
  \def\tableofcontents{\@ifnextchar[{\origtableofcontents}{\gobbletableofcontents}}
  \def\gobbletableofcontents#1{\origtableofcontents}
}

%%%%%%%%%%%%%%%%%%%%%%%%%%%%%% User specified LaTeX commands.
\usetheme{Warsaw}
% or ...

\setbeamercovered{transparent}
% or whatever (possibly just delete it)

\makeatother

\usepackage{babel}
\begin{document}
\title[Short Paper Title]{Title As It Is In the Proceedings}
\subtitle{Include Only If Paper Has a Subtitle}
\author[Author, Another]{F.~Author\inst{1} \and S.~Another\inst{2}}
\institute[Universities of Somewhere and Elsewhere]{\inst{1}Department of Computer Science\\
University of Somewhere\and \inst{2}Department of Theoretical Philosophy\\
University of Elsewhere}
\date[CFP 2003]{Conference on Fabulous Presentations, 2003}

\makebeamertitle

%\pgfdeclareimage[height=0.5cm]{institution-logo}{institution-logo-filename}
%\logo{\pgfuseimage{institution-logo}}

\AtBeginSubsection[]{%
  \frame<beamer>{ 
    \frametitle{Outline}   
    \tableofcontents[currentsection,currentsubsection] 
  }
}

%\beamerdefaultoverlayspecification{<+->}
\begin{frame}{Outline}

\tableofcontents{}

\end{frame}

\section{Motivation}

\subsection[Basic Problem]{The Basic Problem That We Studied}
\begin{frame}{Make Titles Informative. Use Uppercase Letters.}

\framesubtitle{Frame subtitles are optional. Use upper- or lowercase letters.}
\begin{itemize}
\item Use Itemize a lot.

\pause{}
\item Use very short sentences or short phrases.

\pause{}
\item These overlays are created using the Pause style.
\end{itemize}
\end{frame}
%
\begin{frame}{Make Titles Informative. }
\begin{itemize}
\item<1-> You can also use overlay specifications to create overlays.
\item<3-> This allows you to present things in any order.
\item<2-> This is shown second.
\end{itemize}
\end{frame}
%
\begin{frame}{Make Titles Informative.}
\begin{block}<1->{}
\begin{itemize}
\item Untitled block.
\item Shown on all slides.
\end{itemize}
\end{block}
\begin{exampleblock}<2->{Some Example Block Title}
\begin{itemize}
\item $e^{i\pi}=-1$.
\item $e^{i\pi/2}=i$.
\end{itemize}
\end{exampleblock}
\end{frame}

\subsection{Previous Work}
\begin{frame}{Make Titles Informative. }
\begin{example}<1->
On first slide. 
\end{example}

\begin{example}<2->
On second slide.
\end{example}

\end{frame}

\section{Our Results/Contribution}

\subsection{Main Results}
\begin{frame}{Make Titles Informative. }
\begin{theorem}
On first slide.
\end{theorem}


\pause{}
\begin{corollary}
On second slide.
\end{corollary}

\end{frame}
%
\begin{frame}{Make Titles Informative. }
\begin{columns}[t]

\column{5cm}
\begin{theorem}<1->
In left column.
\end{theorem}


\column{5cm}
\begin{corollary}<2->
In right column.\\
New line
\end{corollary}

\end{columns}

\end{frame}

\subsection{Basic Ideas for Proofs/Implementations}

\section*{Summary}
\begin{frame}{Summary}
\begin{itemize}
\item The \alert{first main message} of your talk in one or two lines.
\item The \alert{second main message} of your talk in one or two lines.
\item Perhaps a \alert{third message}, but not more than that.
\end{itemize}
\medskip{}

\begin{itemize}
\item Outlook
\begin{itemize}
\item What we have not done yet.
\item Even more stuff.
\end{itemize}
\end{itemize}
\end{frame}

\appendix

\section*{Appendix}

\subsection*{For Further Reading}
\begin{frame}[allowframebreaks]{For Further Reading}

\beamertemplatebookbibitems
\begin{thebibliography}{1}
\bibitem{Author1990}A. Author. \newblock\emph{Handbook of Everything}.\newblock
Some Press, 1990.\beamertemplatearticlebibitems

\bibitem{Someone2002}S. Someone.\newblock On this and that\emph{.}
\newblock\emph{Journal on This and That}. 2(1):50\textendash 100,
2000.
\end{thebibliography}
\end{frame}

\end{document}
