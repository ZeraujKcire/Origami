
% === DATOS === (((
\title{Origami, Teselaciones y Teoría de Conjuntos.}
\author{Erick Rodríguez.}
% \institute{U.A.A.}
\date{}
% )))

% === PAQUETES === (((
% \usepackage{wasysym}
\usetheme{Warsaw}
\usepackage{amsfonts}
\usepackage{amsmath}
\usepackage{amssymb}
% \usepackage{expl3}
\usepackage{tikz}
\usepackage{mathrsfs}
\usepackage{graphicx}
% )))

% === TIPOGRAFÍA === (((
\usefonttheme{professionalfonts}
\usefonttheme{serif}
\usepackage{fontspec}
\setmainfont[
  BoldFont       = bodonibi,
	ItalicFont     = Century modern italic2.ttf,
	BoldItalicFont = bodonibi,
	SmallCapsFont  = lmromancaps10-regular.otf
]{Century_modern.ttf}
% \usepackage{expl3}
\DeclareSymbolFont{italics}{\encodingdefault}{\rmdefault}{m}{it}
\DeclareSymbolFontAlphabet{\mathit}{italics}
\ExplSyntaxOn
\int_step_inline:nnnn { `A } { 1 } { `Z }
 {  \exp_args:Nf \DeclareMathSymbol{\char_generate:nn{#1}{11}}{\mathalpha}{italics}{#1} }
\int_step_inline:nnnn { `a } { 1 } { `z } {  \exp_args:Nf \DeclareMathSymbol{\char_generate:nn{#1}{11}}{\mathalpha}{italics}{#1}}
\ExplSyntaxOff
% )))

% === COMANDOS === (((
\newcommand{\dis}{\displaystyle}
\renewcommand{\qed}{\hspace{0.5cm}\rule{0.16cm}{0.4cm}}
\newcommand\Myref[1]{
  \begingroup
  \usebeamerfont*{item projected}%
  \usebeamercolor[bg]{item projected}%
  \begin{pgfpicture}{-1ex}{0ex}{1ex}{2ex}
    \pgfpathcircle{\pgfpoint{0pt}{.75ex}}{1.2ex}
    \pgfusepath{fill}
    \pgftext[base]{\color{fg}\ref{#1}}
  \end{pgfpicture}%
  \endgroup
}
\newcommand{\operator}[1]{\mathop{\vphantom{\sum}\mathchoice
{\vcenter{\hbox{\huge $#1$}}}
{\vcenter{\hbox{\Large $#1$}}}{#1}{#1}}\displaylimits}
\newcommand{\suma}{\operator{\includegraphics[scale=0.09]{IMAGENES/Sigma.png}}}
\setlength{\parindent}{0mm}
% )))

% === BEAMER TEMPLATE === (((
% TITULOS EN SMALL CAPS
\setbeamerfont{frametitle}{family=\scshape\LARGE}
\setbeamerfont{section in head/foot}{size = \scriptsize}
% sin footline
\setbeamertemplate{footline}{}
% enumerates sin shade
\setbeamertemplate{enumerate items}[circle] % esto lo estoy usando para los simbolos de enumeración.
% items sin shade
\setbeamertemplate{itemize item}{\tikz \fill [MSUgreen] (0,0) circle (0.5ex);}
\setbeamertemplate{itemize subitem}{\tikz \fill [MSUgreen] (0,0) circle (0.5ex);}
% quitar espacio
\addtobeamertemplate{titleframe}{}{\vspace{-5mm}}
\addtobeamertemplate{block begin}{\vskip - \bigskipamount}{}
\addtobeamertemplate{block end}{}{\vskip - \smallskipamount}
\addtobeamertemplate{block example begin}{\vskip - \bigskipamount}{}
\addtobeamertemplate{block example end}{}{\vskip - \smallskipamount}
% PARA VER SECCIONES HORIZONTALMENTE
\setbeamertemplate{headline}{
\leavevmode
\hbox{
	\begin{beamercolorbox}[wd=\paperwidth,ht=4ex,dp=2ex]{palette quaternary}
\insertsectionnavigationhorizontal{\paperwidth}{\hskip 0pt plus1filll}{\hskip 0pt plus1filll}
\end{beamercolorbox}
% 
}
% 
}
% SECCIONES EN PÁGINAS.
\setbeamerfont{section title}{parent=title}
\setbeamertemplate{section page}
{
    \begin{centering}
	    \begin{beamercolorbox}[sep=12pt,center]{secciones}
    \usebeamerfont{frametitle}\insertsection\par
    \end{beamercolorbox}
    \end{centering}
}
% )))

% === COLORS === (((
\beamertemplatenavigationsymbolsempty % for remove the nav. symb.
\mode<presentation>

\definecolor{MSUgreen}{RGB}{216, 90, 55}
% rgb(216, 90, 55)

\setbeamercolor{alerted text}{fg=black}
\setbeamercolor*{palette primary}{fg=white,bg=white!60!MSUgreen}
\setbeamercolor*{palette secondary}{fg=black,bg=white!40!MSUgreen}
\setbeamercolor*{palette tertiary}{bg=black,fg=white!30!MSUgreen}
\setbeamercolor*{palette quaternary}{fg=MSUgreen!10!white,bg=blue!10!MSUgreen}

\setbeamercolor*{sidebar}{fg=MSUgreen,bg=MSUgreen!75!white}

% \setbeamercolor*{palette sidebar primary}{fg=MSUgreen!10!black}
% \setbeamercolor*{palette sidebar secondary}{fg=white!80!blue}
% \setbeamercolor*{palette sidebar tertiary}{fg=MSUgreen!50!black}
% \setbeamercolor*{palette sidebar quaternary}{fg=black!80!MSUgreen}

\setbeamercolor*{titlelike}{parent=palette primary}
\setbeamercolor{frametitle}{bg=blue!10!MSUgreen}
\setbeamercolor{frametitle right}{bg=white!60!MSUgreen}

\setbeamercolor*{separation line}{}
\setbeamercolor*{fine separation line}{}
\setbeamercolor{block body}{parent=normal text,use=block title,bg=MSUgreen!15!white,fg=black}
\setbeamercolor{block title}{bg=MSUgreen!90!blue,fg=white}

\setbeamercolor{block title example}{bg=MSUgreen!40!blue,fg=white}
\setbeamercolor{block body example}{parent=normal text,use=block title,bg=MSUgreen!15!white,fg=black}
\setbeamercolor{item projected}{bg=MSUgreen}

\setbeamercolor{section title}{parent=titlelike}
\setbeamercolor{secciones}{fg=black,bg=orange!20}
% \setbeamercolor{section in head}{fg=white, bg = blue!50!MSUgreen}
% \setbeamertemplate{section in head/foot shaded}{\color{gray}\usebeamertemplate{section in head/foot}}
\mode
<all>
% )))
