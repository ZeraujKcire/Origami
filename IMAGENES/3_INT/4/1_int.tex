\documentclass[a4paper, 10pt]{article}
\usepackage[left=2.5cm,top=2.5cm,right=2.5cm,bottom=2.5cm]{geometry}

\usepackage[utf8]{inputenc}
\usepackage{ragged2e}  
\usepackage{cancel}
\usepackage{mathrsfs}
\usepackage{amsmath}
\usepackage{graphicx}
\usepackage{array}
\usepackage{hyperref}
% \usepackage[spanish, es-tabla]{babel}
% \spanishdecimal{.}
\usepackage{multicol}
\usepackage{multirow}
\usepackage{amssymb}
\usepackage[usenames,dvipsnames,svgnames,table]{xcolor}
\usepackage{dsfont}
\usepackage{parskip}
\usepackage{booktabs}
\usepackage{listings}
\usepackage{subfig}
\usepackage[table]{xcolor}
% \usepackage{slashbox}
\usepackage{tikz}
\usetikzlibrary{shapes.geometric}
\usepackage{tkz-euclide}

\begin{document}
	
	\textbf{\textit{\underline{Def.:}}}
	
	El conjunto de dobleces posibles es $ D=\{a_i, b_j, c_k | i,j\in J_{17}, k\in J_{30} \} $, un conjunto de segmentos de rectas. 
	
	\textbf{\textit{\underline{Obs.:}}}
	
	Los dobleces que se harán al manifold son elementos del conjunto $ D-\{a_1,b_1,a_{17},b_{17}\} $.
	
	\textbf{\textit{\underline{Obs.:}}}
	
	Se ocupan 16 dobleces horizontales y verticales por la configuración 5 y similares.
	
	\textbf{\textit{\underline{Obs.:}}}
	
	Todo triángulo de una configuración queda atrapado por exactamente \underline{3 dobleces} (trivialmente).
	
	\textbf{\textit{\underline{Def.:}}}
	
	Definimos el conjunto de triángulos de un manifold 
	$$ M=\left\{ (a_i,c_j,c_k), (b_i,c_j,c_k), i\in J, j\in J,k=16..., 
	a_i \cap c_j\neq D, a_i\cap c_j\neq\phi, 
	c_j \cap c_k\neq D, b_i\cap c_j\neq\phi, 
	b_i \cap c_k\neq\phi \right\} $$
	
	\textbf{\textit{\underline{Obs.:}}}
	
	$ M\subseteq D^3 $
	
	\textbf{\textit{\underline{Def.:}}}
	
	Consideremos el conjunto de vértices $ V=M $ y hacemos que para cada par de vértices adyacentes exista \underline{una arista}
	
	$ |\pi_1(b)\cap\pi_1(e)|>1 $
	
	$ |\pi_i(a)\cap\pi_i(e)|\leqslant 1 $
	
	$ |\pi_1(c)\cap\pi_1(i)|>1 $
	
	$ |\pi_i(e)\cap\pi_i(f)|\leqslant 1 $
	
	$ |\pi_3(a)\cap\pi_3(d)|>1 $
	
	$ |\pi_i(b)\cap\pi_i(d)|\leqslant 1 $
	
	$ |\pi_i(a)\cap\pi_i(k)|\leqslant 1 $
	
	$ |\pi_3(b)\cap\pi_3(c)|>1 $
	
	$ |\pi_2(c)\cap\pi_2(d)|>1 $
	
	\begin{align*}
		E&=\left\{ \{u, v\}: |\pi_i(u)\cap\pi_i(v)|>1,\mbox{ para algún }i \right\} \\
		&\subseteq\left\{ \{ u,v \}: u,v\in M \right\}
	\end{align*}
	
	ese el grafo $ G=(V,E) $ que describe la forma de todos los manifolds. 
	
	\textbf{\textit{\underline{Def.:}}}
	
	Sea $ G=(V,E) $ el grafo anterior. Definimos una configuración de un \underline{manifold} como una función $ c:V\longrightarrow J_3 $ donde los elementos de $ J_3 $ representan lo siguiente:
	\begin{itemize}
		\item [a)] El color 1 se interpreta como blanco
		
		\item [b)] El color 2 se interpreta como negro
		
		\item [c)] El color 3 se interpreta como transparente
	\end{itemize}
	
	tal que $ c=1 $ en 32 triángulos, $ c=2 $ en 32 y $ c=3 $ en 64. 
	
	\newpage
	ALGORITMO
	
	1. Se tiene el grafo coloreado
	
	\textbf{\textit{\underline{Obs.:}}}
	
	del mismo color representan unión de triángulos
	
	2. Identificar todos los triángulos de cada color.
	
	3. Identificar todos los triángulos de un mismo color conectado.
	
	\textbf{\textit{\underline{Def.:}}}
	
	En un grafo $ G=(V,E) $ decimos que $ u\in V $ es \underline{vecino} de $ v\in V $ si $ \exists(u,v)\in E $. decimos que $ u,v $ son adyacentes
	
	\textbf{\textit{\underline{Def.:}}}
	
	Decimos que el grafo $ G $ es \underline{conexo} si $ \forall u\in V, \exists v\in V $ tal que $ u $ y $ v $ son vecinos 
	
	\textbf{\textit{\underline{Def.:}}}
	
	Dado un grafo $ G=(V,E) $, un \underline{subgrafo} $ G' $ es un par $ V'\subseteq U \wedge E'\subseteq V' \times V' $ con $ G'=(V', E') $.
	
	\textbf{\textit{\underline{Def.:}}}
	
	Un \underline{área} o \underline{región} de un \underline{manifold} es un subgrafo \fbox{conexo} $ G'\subseteq G $ tal que $ c(G') $ tiene un solo elemento.
	
	\textbf{\textit{\underline{Obs.:}}}
	
	Todos los nodos de un área tienen el mismo color.
	
	\textbf{\textit{\underline{Def.:}}}
	
	Un \underline{nodo frontera} es un $ a\in M $ tal que $ c(a)\neq c(b) $ para algún nodo $ b\in M $ adyacente a $ a $
	
	\textbf{\textit{\underline{Def.:}}}
	
	Definimos
	
	$ B=\{ b\in M | c(b)=1 \} $, 
	
	$ N=\{ n\in M | c(n)=2 \} $, 
	
	$ T=\{ t\in M | c(t)=3 \} $
	
	\textbf{\textit{\underline{Obs.:}}}
	
	$ B $ es el conjunto de triángulos blancos, $ N $ los negros y $ T $ los transparentes.
	
	\textbf{\textit{\underline{Obs.:}}}
	$ \{B, N, T\} $ es una partición de $ M $
	
	\textbf{\textit{\underline{Def.:}}}
	
	Denotaremos como
	
	$ E_1=\left\{ \{u, v\}: u,v\in B \wedge |\pi_i(u)\cap\pi_i(v)|>1,\mbox{ para algún }i \right\}  $
	
	$ E_2=\left\{ \{u, v\}: u,v\in N \wedge |\pi_i(u)\cap\pi_i(v)|>1,\mbox{ para algún }i \right\}  $
	
	$ E_3=\left\{ \{u, v\}: u,v\in T \wedge |\pi_i(u)\cap\pi_i(v)|>1,\mbox{ para algún }i \right\}  $
	
	
	\textbf{\textit{\underline{Obs.:}}} 
	
	Para resolver el problema hay que considerar las componentes conexas de los grafos
	$$ G_1=(B, E_1), G_2=(N, E_2) G_3=(T, E_3) $$
	y sus fronteras.
	
	\textbf{\textit{\underline{Def.:}}}
	
	Un \underline{área máximal} es un área de $ G $ que es una componente conexa de $ G_i $ para algún $ i $.
		
	4. identificar las áreas maximales y sus fronteras
	
	5. trazar los dobleces de todas las fronteras.
	
\end{document}
